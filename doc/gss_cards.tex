% This is the manual for the GSS 0.46.
%
% Copyright (C) 2005, 2006 Gong Ding.
% Email : gdiso@ustc.edu
% Permission is granted to copy, distribute and/or modify this document
% under the terms of the GNU Free Documentation License, Version 1.1 or
% any later version published by the Free Software Foundation; with no
% Invariant Sections, with no Front-Cover Texts, and with no Back-Cover
% Texts.

\RequirePackage{ifpdf}
\ifpdf % We are running pdfTeX in pdf mode
\documentclass[11pt,pdftex]{article}
\else
\documentclass{article}
\fi

% Page layout.
\advance\textwidth by 1.1in
\advance\oddsidemargin by -.55in
\advance\evensidemargin by -.55in
%
\advance\textheight by 1in
\advance\topmargin by -.2in
\advance\footskip by -.2in
%
\pagestyle{headings}
%
% Avoid some overfull boxes.
\emergencystretch=.1\hsize \hbadness = 3000

% these are from lshort.sty, but lshort.sty pulls in so many other
% packages it seems cleaner to just include them here.
%
\newcommand{\bs}{\symbol{'134}}%Print backslash
\newcommand{\ci}[1]{\texttt{\bs#1}}

\makeatletter
\@ifpackageloaded{tex4ht}{%
  % separate definition for HTML case to avoid
  % nasty borders with double horizontal lines with
  % large gaps.
  \newsavebox{\cmdsyntaxbox}%
  \newenvironment{cmdsyntax}{%
    \par
    % \small
    \addvspace{3.2ex plus 0.8ex minus 0.2ex}%
    \vskip -\parskip
    \noindent
    \begin{lrbox}{\cmdsyntaxbox}%
      \begin{tabular}{l}%
        \rule{0pt}{1em}%
        \ignorespaces
  }{%
      \end{tabular}%
    \end{lrbox}%
    \fbox{\usebox{\cmdsyntaxbox}}%
    \par
    \nopagebreak
    \addvspace{3.2ex plus 0.8ex minus 0.2ex}%
    \vskip -\parskip
  }%
}{%
  \newenvironment{cmdsyntax}{%
    \par
    \small
    \addvspace{3.2ex plus 0.8ex minus 0.2ex}%
    \vskip -\parskip
    \noindent
    \begin{tabular}{|l|}%
      \hline
      \rule{0pt}{1em}%
      \ignorespaces
  }{%
      \\%
      \hline
    \end{tabular}%
    \par
    \nopagebreak
    \addvspace{3.2ex plus 0.8ex minus 0.2ex}%
    \vskip -\parskip
  }%
} \makeatother

\usepackage[pdftex]{graphicx}
\usepackage{flafter}
\usepackage{array,indentfirst,longtable,amsmath}
\usepackage[T1]{fontenc}


\def\Hanh{H\`an Th\^e\llap{\raise 0.5ex\hbox{\'{}}} Th\`anh}

\ifpdf
  \usepackage[%
    pdftex,%
    colorlinks,%
    hyperindex,%
    plainpages=false,%
    bookmarks=true,%
    bookmarksnumbered=true%
  ]{hyperref}
  %%?? \def\pdfBorderAttrs{/Border [0 0 0] } % No border arround Links
  \usepackage{thumbpdf}
\else
  \usepackage{hyperref}
\fi

\pdfinfo{
   /Title  (GSS User's Guide)
   /Subject(User's Guide of GSS-0.46)
   /Author (Gong Ding, gdiso@ustc.edu)
   /Keywords (GSS, TCAD, Semiconductor Simulation, DDM, HDM)
   /CreationDate (D:20061116120000)
   /ModDate (D:20070219120000)
}

\title{GSS User's Guide Ver 0.46.07}
\author{Gong Ding
    \\University of Science and Technology of China
    \\\centerline{\includegraphics[scale=0.3]{USTCLogo.png}}
    \\Email: gdiso@ustc.edu
}


\begin{document}

% comes out too close to the toc, and we know it's page one anyway.
\thispagestyle{empty}
\maketitle
\tableofcontents
\setcounter{tocdepth}{2}% for bookmark levels

\section{Introduction}
\subsection{The format of input card}
Like PISCES and MEDICI, GSS takes its command cards from a user
specified disk file. The input is read by GSS's build-in command
parser. Each line is recognized as a particular statement,
identified by the first word (named as keyword) on the card. The
remaining parts of the line are the parameters of that keyword. The
statement has the format as follow:
\begin{verbatim}
            KEYWORD [parameters]
\end{verbatim}
The words on a line are separated by blanks or tabs. If more than
one line of input is necessary for a particular statement, it may be
continued on subsequent lines by placing a backslash sign
'\textbackslash' as the last non-blank character on the current
line. Parameters may be one of four types: float, integer, bool or
string. The float point number supports C style double precision
real number. The bool value can be True, On, False and Off. String
value is made up of lower line, dot, blank, number and alpha
characters. The string should not begin with number and quotation
marks are only needed if it contains blank. At last, the length of
string is limited to 31 characters. All the parameter specification
has the same format as
\begin{verbatim}
            parameter_name = [number|integer|bool|string]
\end{verbatim}


In the card descriptions, keywords and parameters are not case
sensitive. But user input strings do, because file name may be
specified by the string. Comments must begin with '\#' and can be
either an separated line or locate at the end of current statement.

\subsection{The sequence of input deck}
Most of the cards GSS used are sequence insensitive. The order of
occurrence of cards is significant in only two cases. The mesh
generation cards must have the right order, or it can't work
properly. GSS will execute the 'driven' cards sequently. So the
placement order of 'driven' cards will affect simulation result.

\subsection{Statement Description Format}
\subsubsection*{Syntax of Parameter Lists}
The following special characters are used in the formatted parameter
list:
\begin{verbatim}
        Angle brackets < >  - parameter type
        Square brackets [ ] - optional group
        Vertical bar |      - alternate choice
        Parentheses ( )     - group hierarchy
        Braces { }          - group hierarchy with high level
\end{verbatim}

\subsubsection*{Value Types}
Besides some string parameters which have fixed values, most of the
parameters need a user defined value. A lower case letter in angle
brackets represents a value of a given type. The following types of
values are represented:
\begin{verbatim}
        <n> - double precision numerical value
        <i> - integer value
        <b> - bool value
        <s> - string value
\end{verbatim}



\newpage
\section{Global Specification}
\subsection{SET}
\subsubsection*{Description} Some global definitions such as the unit scale
and environment temperature must be set before the initiation of
GSS's build-in data. The \textsf{SET} command will do the
definition.

\subsubsection*{Syntax}

\begin{verbatim}
        set Carrier=(p|n|pn)
        set Z.Width=<n>
        set LatticeTemp=<n>
        set DopingScale=<n>
\end{verbatim}

\small
\noindent\begin{longtable}{ccccp{7cm}}
\textbf{parameter}   & \textbf{type}          & \textbf{default} & \textbf{unit} & \textbf{description} \\
Carrier     & string  & pn   & -              & The \textbf{Carrier} parameter specifies whether single or dual
                                                carriers will be modeled during the simulation. But at present,
                                                GSS only supports dual carriers, so the parameter value must always be "pn" \\
Z.Width     & number  & 1    & $\mathrm{\mu m}$ & \textbf{Z.Width} is needed by current calculation. Because GSS
                                                is a two-dimensional simulator, the length in Z direction must be
                                                given if GSS simulates transistor with external circuit.  \\
LatticeTemp & number  & 300  & $\mathrm{K}$        & \textbf{LatticeTemp} defines external temperature.      \\
DopingScale & number  & 1e18 & $\mathrm{cm^{-3}}$  & \textbf{DopingScale} will effect GSS's inner unit scale procedure
                                                     which shows great influence to the convergence of nonlinear solver.
                                                     In most case, set this value to max(Nd,Na) is a good choice. But
                                                     sometimes, a smaller value may be better.\\
\end{longtable}
\normalsize

\subsubsection*{Example}
\begin{verbatim}
        set Carrier     = pn    # specify carrier type.
        set Z.Width     = 2     # device width in Z dimension. Unit:um
        set LatticeTemp = 3e2   # specify initial temperature of device. Unit:K
        set DopingScale = 1e16  # set carrier scale reference value
\end{verbatim}


\newpage
\section{Mesh Generation}
\subsection{Introduction}
The early version of GSS was designed as a pure solver. It uses
CGNS(CFD General Notation System) as semiconductor device model
file. This file format provides the ability to store grid, solution
data, material information, boundary condition and connectivity in a
single, well-defined and easy-to-use form. More important, CGNS has
been accepted and supported by most of the commercial CFD
corporations. So users have various ways to create their models. For
example, models can be created by SGFramework, converted from MEDICI
TIF file by TIFTOOL (shipped with GSS) or generated by ICEMCFD,
which is a commercial CFD pre-processor.

Until very recently, the PISCES like model description language had
been introduced to GSS. The mesh generation arithmetic works as
follows. First, GSS builds the rectangle skeleton mesh by the model
description statements; Then, GSS employs Triangle (developed by
Jonathan Richard Shewchuk) to form the triangulate mesh and output
the mesh to an initial CGNS file. At last, GSS reads the CGNS file
again, computes the doping profile and finishes the remaining
calculations.

Triangle uses delaunay arithmetic, which forms a high quality
isotropic mesh. At the same time, MEDICI uses quadtree arithmetic to
generate its mesh, which often gives a regular mesh but the mesh
quality may be poor near the irregular boundary.

\subsection{Coordinate System}
The mesh generator uses a Cartesian coordinate system, in which the
top horizontal line has the maximal y coordinate and left vertical
line has the minimal x coordinate.

Note: This setting is different from PISCES and its commercial
versions like MEDICI and ATLAS.

\newpage
\subsection{MESH}
This statement indicates the beginning of the mesh generator.
\subsubsection*{Syntax}
\begin{verbatim}
        MESH [ Type=<s> ] ModelFile=<s> [ Triangle=<s> ]
\end{verbatim}

\small
\noindent\begin{longtable}{ccccp{7cm}}
\textbf{parameter}   & \textbf{type}  & \textbf{default} & \textbf{unit} & \textbf{description} \\
Type        & string  & -    & -      & \textbf{Type} indicates which mesh generator is to be used.
                                        But at present it is useless since GSS only has one mesh generator. \\
ModelFile   & string  & -    & -      & \textbf{ModelFile} gives the name of temporary CGNS file. \\
Triangle    & string  & pzq30AD  & -  & \textbf{Triangle} passes parameters to Triangle code.
                                        The detailed description of this string can be found at Triangle's home page
                                        http://www.cs.cmu.edu/~quake/triangle.html.\\
\end{longtable}
\normalsize

\subsubsection*{Example}
\begin{verbatim}
            MESH Type=GSS ModelFile=pn.cgns Triangle="pzA"
\end{verbatim}


\newpage
\subsection{XMESH and YMESH}
The \textbf{XMESH} and \textbf{YMESH} cards specify the location of
lines of nodes in a rectangular mesh. The original mesh can be
modified by following mesh cards like \textbf{ELIMINATE} and
\textbf{SPREAD}.

\subsubsection*{Syntax}
\begin{verbatim}
        XMESH  { WIDTH=<n> | ( X.MIN=<n> X.MAX=<n> ) }
               { N.SPACES=<i> [RATIO=<n>] | H1=<n> [H2=<n>] }
        YMESH  { DEPTH=<n> | ( Y.MAX=<n> Y.MIN=<n> ) }
               { N.SPACES=<i> [RATIO=<n>] | H1=<n> [H2=<n>] }
\end{verbatim}

\small
\noindent\begin{longtable}{ccccp{7cm}}
\textbf{parameter}   & \textbf{type}  & \textbf{default} & \textbf{unit} & \textbf{description} \\
WIDTH       & number  & -    & $\mathrm{\mu m}$     & The distance of the grid section in x direction. \\
DEPTH       & number  & -    & $\mathrm{\mu m}$     & The distance of the grid section in y direction. \\
X.MIN       & number  & -    & $\mathrm{\mu m}$     & The x location of the left edge of the grid section.
                                                      synonym: \textbf{X.LEFT}. The value of \textbf{X.MIN} will be set to right
                                                      edge of the previous grid section automatically.\\
X.MAX       & number  & -    & $\mathrm{\mu m}$     & The x location of the right edge of the grid section.
                                                      synonym: \textbf{X.RIGHT}. \\
Y.MIN       & number  & -    & $\mathrm{\mu m}$     & The y location of the bottom edge of the grid section.
                                                      synonym: \textbf{Y.BOTTOM}. \\
Y.MAX       & number  & -    & $\mathrm{\mu m}$     & The y location of the top edge of the grid section.
                                                      synonym: \textbf{Y.TOP}. The value of \textbf{Y.MAX} will be set to bottom
                                                      edge of the previous grid section automatically.\\
N.SPACES    & integer & 1    & -                    & The number of grid spaces in the grid section.\\
RATIO       & number  & 1.0  & -                    & The ratio between the sizes of adjacent grid spaces in the grid
                                                      section. \textbf{RATIO} should usually lie between 0.667 and 1.5. \\
H1          & number  & -    & $\mathrm{\mu m}$     & The size of the grid space at the begin edge of the grid section.\\
H2          & number  & -    & $\mathrm{\mu m}$     & The size of the grid space at the end edge of the grid section.\\
\end{longtable}
\normalsize

\subsubsection*{Example}
\begin{verbatim}
            XMESH    X.MIN=0.0  X.MAX=0.50   N.SPACES=8
            YMESH    DEPTH=0.1  N.SPACES=8   RATIO=0.8
            YMESH    DEPTH=0.1  N.SPACES=20
            YMESH    DEPTH=0.6  H1=0.005  H2=0.050
\end{verbatim}


\newpage
\subsection{ELIMINATE}
The \textbf{ELIMINATE} statement eliminates mesh points along planes
in a rectangular grid over a specified volume. This statement is
useful for eliminating nodes in regions of the device structure
where the grid is more dense than necessary. Points along every
second line in the chosen direction within the chosen range are
removed, except the first and last line. Successive eliminations of
the same range remove points along every fourth line, eighth line,
and so on.

\subsubsection*{Syntax}
\begin{verbatim}
        ELIMINATE   { DIRECTION = (ROWS | COLUMNS) }
                    [ {X.MIN=<n> | IX.MIN=<i>} ] [ {X.MAX=<n> | IX.MAX=<i>} ]
                    [ {Y.MIN=<n> | IY.MAX=<i>} ] [ {Y.MAX=<n> | IY.MIN=<i>} ]
\end{verbatim}

\small
\noindent\begin{longtable}{ccccp{7cm}}
\textbf{parameter}   & \textbf{type}  & \textbf{default} & \textbf{unit} & \textbf{description} \\
DIRECTION   & string  & -       & -                    & Specifies that horizontal or vertical lines of nodes are eliminated. \\
X.MIN       & number  & XMIN    & $\mathrm{\mu m}$     & The minimum x location of the rectangular volume in which nodes are eliminated.
                                                         synonym: \textbf{X.LEFT}. \\
X.MAX       & number  & XMAX    & $\mathrm{\mu m}$     & The maximum x location of the rectangular volume in which nodes are eliminated.
                                                         synonym: \textbf{X.RIGHT}. \\
IX.MIN      & integer & 0       & -                    & The minimum x node index of the rectangular volume in which nodes are eliminated.
                                                         synonym: \textbf{IX.LEFT}.\\
IX.MAX      & integer & IXMAX-1 &-                     & The maximum x node index of the rectangular volume in which nodes are eliminated.
                                                         synonym: \textbf{IX.RIGHT}. \\
Y.MIN       & number  & YMIN    & $\mathrm{\mu m}$     & The minimum y location of the rectangular volume in which nodes are eliminated.
                                                         synonym: \textbf{Y.BOTTOM}. \\
Y.MAX       & number  & YMAX    & $\mathrm{\mu m}$     & The maximum y location of the rectangular volume in which nodes are eliminated.
                                                         synonym: \textbf{Y.TOP}. \\
IY.MIN      & integer & 0       & -                    & The minimum y node index of the rectangular volume in which nodes are eliminated.
                                                         synonym: \textbf{IY.TOP}. \\
IY.MAX      & integer & IYMAX-1 & -                    & The maximum y node index of the rectangular volume in which nodes are eliminated.
                                                         synonym: \textbf{IY.BOTTOM}.
\end{longtable}
\normalsize

\subsubsection*{Example}
\begin{verbatim}
            ELIMINATE    Direction=COLUMNS  Y.TOP=-1.0
            ELIMINATE    Direction=ROWS     IX.MAX=8
\end{verbatim}


\newpage
\subsection{SPREAD}
The \textbf{SPREAD} statement provides a way to adjust the y
position of nodes along grid lines parallel to the x-axis in a
rectangular mesh to follow surface and junction contours.

\subsubsection*{Syntax}
\begin{verbatim}
       SPREAD LOCATION=(LEFT|RIGHT) WIDTH=<n> UPPER=<i> LOWER=<i> [ENCROACH=<n>]
              { Y.LOWER=<n>  | (THICKNES=<n> [VOL.RAT=<n>]) }
              [GRADING=<n>]
\end{verbatim}

\small
\noindent\begin{longtable}{ccccp{7cm}}
\textbf{parameter}   & \textbf{type}  & \textbf{default} & \textbf{unit} & \textbf{description} \\
LOCATION    & string  & -       & -                    & Specifies which side of the grid is distorted. \\
WIDTH       & number  & 0.0     & $\mathrm{\mu m}$     & The width of the distorted region measured from the left or
                                                         right edge of the structure. \\
UPPER       & integer & 0       & -                    & The index of the upper y-grid line of the distorted region. \\
LOWER       & integer & 0       & -                    & The index of the lower y-grid line of the distorted region. \\
ENCROACH    & number  & 1.0     & -                    & The factor which defines the abruptness of the transition
                                                         between distorted and undistorted grid. The transition region
                                                         becomes more abrupt with smaller \textbf{ENCROACH} factors. The minimum
                                                         allowed value is 0.1. \\
Y.LOWER     & number & -        & $\mathrm{\mu m}$     & The vertical location in the distorted region where the line
                                                         specified by \textbf{LOWER} is moved. The grid line specified by
                                                         \textbf{UPPER} does not move if this parameter is specified. \\
THICKNESS   & number & -        & $\mathrm{\mu m}$     & The thickness of the distorted region. Specifying \textbf{THICKNESS}
                                                         usually causes the positions of both the \textbf{UPPER} and \textbf{LOWER}
                                                         grid lines to move. \\
VOL.RAT     & number & 0.44     & -                    & The ratio of the displacement of the lower grid line to the net
                                                         change in thickness. If \textbf{VOL.RAT} is 0, the location of the
                                                         lower grid line does not move. If \textbf{VOL.RAT} is 1, the upper
                                                         grid line does not move. \\
GRADING     & number  & 1.0      & -                   & The vertical grid spacing ratio in the distorted region between
                                                         the y-grid lines specified with \textbf{UPPER} and \textbf{LOWER}
                                                         The spacing grows or shrinks by
                                                         GRADING in each interval between lines. \textbf{GRADING} should
                                                         usually lie between 0.667 and 1.5.
\end{longtable}
\normalsize

\subsubsection*{Example}
\begin{verbatim}
            SPREAD    Location=Left  Width=0.625 Upper=0 Lower=2 Thickness=0.1
            SPREAD    Location=Right Width=0.625 Upper=0 Lower=2 Thickness=0.1
\end{verbatim}


\newpage
\subsection{REGION}
The \textbf{REGION} statement defines the location of materials in
the mesh. Currently, GSS supports following materials: null space
including Vacuum and Air; semiconductor material including Si, Ge,
GaAs, $\mathrm{Si}_{1-x}\mathrm{Ge}_x$,
$\mathrm{Al}_x\mathrm{Ga}_{1-x}\mathrm{As}$ and
$\mathrm{In}_x\mathrm{Ga}_{1-x}\mathrm{As}$; insulator material
including $\mathrm{SiO}_2$ and electrode region including Elec, Al
and PolySi.

\subsubsection*{Syntax}
\begin{verbatim}
   REGION Shape=Rectangle Label=<s> Material=<s>
          [ X.MOLE=<n> [ MOLE.SLOPE=<n> | MOLE.END=<n> ] MOLE.GRAD=(X.Linear|Y.Linear) ]
          [ {X.MIN=<n> | IX.MIN=<n>} ] [ {X.MAX=<n> | IX.MAX=<n>} ]
          [ {Y.MIN=<n> | IY.MIN=<n>} ] [ {Y.MAX=<n> | IY.MAX=<n>} ]
   REGION Shape=Ellipse Label=<s> Material=<s>
          {CentreX=<n> CentreY=<n> MajorRadii=<n> MinorRadii=<n> Theta=<n> Division=<i>}
\end{verbatim}

\small
\noindent\begin{longtable}{ccccp{8.5cm}}
\textbf{parameter}   & \textbf{type}  & \textbf{default} & \textbf{unit} & \textbf{description} \\
Shape       & string  & -       & -                    & Specifies the shape of the region. Can be Rectangle or Ellipse.\\
Label       & string  & -       & -                    & Specifies the identifier of this region, limited to 12 chars. \\
Material    & string  & -       & -                    & Specifies the material of the region. Material strings can be
                                                         Vacuum, Air, Si, Ge, GaAs, SiGe, AlGaAs, InGaAs, SiO2, Elec, Al and PolySi.\\
X.MOLE      & number  & 0.0     & -                    & The mole fraction to use in the region for compound materials.
                                                         For graded compounds, \textbf{X.MOLE} represents the initial
                                                         mole fraction at the left, top, or front edge of the region
                                                         depending on whether X.Linear, or Y.Linear, respectively,
                                                         is specified. \\
MOLE.SLOPE & number   & 0.0     & $\mathrm{\mu m}^{-1}$ & The slope of the mole fraction for graded compounds. If this
                                                         parameter is used, the mole fraction has a value of \textbf{X.MOLE}
                                                         at the left, top or front edge of the region and a value of
                                                         \textbf{X.MOLE} + width * \textbf{X.SLOPE} at the right, bottom or back
                                                         edge of the region, where width is the width or depth of the
                                                         region. \\
MOLE.END   & number   & 0.0     & -                     & The mole fraction for graded compounds at the right, bottom,
                                                          or backedge of the region depending on whether X.Linear,
                                                          or Y.Linear, respectively, is specified. \\
MOLE.GRAD  & string   & Y.Linear & -                    & Specifies that the mole fraction grading is in the x or y direction. \\
X.MIN       & number  & XMIN    & $\mathrm{\mu m}$     & The minimum x location of the region.
                                                         synonym: \textbf{X.LEFT}. \\
X.MAX       & number  & XMAX    & $\mathrm{\mu m}$     & The maximum x location of the region.
                                                         synonym: \textbf{X.RIGHT}. \\
IX.MIN      & integer & 0       & -                    & The minimum x node index of the region.
                                                         synonym: \textbf{IX.LEFT}.\\
IX.MAX      & integer & IXMAX-1 &-                     & The maximum x node index of the region.
                                                         synonym: \textbf{IX.RIGHT}. \\
Y.MIN       & number  & YMIN    & $\mathrm{\mu m}$     & The minimum y location of the region.
                                                         synonym: \textbf{Y.BOTTOM}. \\
Y.MAX       & number  & YMAX    & $\mathrm{\mu m}$     & The maximum y location of the region.
                                                         synonym: \textbf{Y.TOP}. \\
IY.MIN      & integer & 0       & -                    & The minimum y node index of the region.
                                                         synonym: \textbf{IY.TOP}. \\
IY.MAX      & integer & IYMAX-1 & -                    & The maximum y node index of the region.
                                                         synonym: \textbf{IY.BOTTOM}.\\
CentreX     & number  & 0.0     & $\mathrm{\mu m}$   & The x location of the center of ellipse. \\
CentreY     & number  & 0.0     & $\mathrm{\mu m}$   & The y location of the center of ellipse.\\
MajorRadii  & number  & 1.0     & $\mathrm{\mu m}$   & The length of the major radii of ellipse.\\
MinorRadii  & number  & MajorRadii     & $\mathrm{\mu m}$   & The length of the minor radii of ellipse.\\
Theta       & number  & 0.0     & degree               & The angle of the first division point located on the boundary of ellipse region.\\
Division    & integer & 12      & -                    & The number of points which divide the boundary of ellipse into small segments.
\end{longtable}
\normalsize

\subsubsection*{Example}
\begin{verbatim}
            REGION    Label=Si1   Material=Si    Y.TOP= 0.000  Y.BOTTOM=-0.100
            REGION    Label=SiGe1 Material=SiGe  Y.TOP=-0.100  Y.BOTTOM=-0.125 \
                      X.MOLE=0.0  Mole.End=0.2
            REGION    Label=Hole  Material=SiO2 Shape=Ellipse CentreX=2.0 CentreY=-0.5 \
                      Division=24 MajorRadii=0.3 MinorRadii=0.3
\end{verbatim}

\subsubsection*{Hint}
Several regions can be defined one by one. But users should be
careful that regions can't get cross each other. The situations
showed by Fig\ref{Region1} (A) and (B) are allowed, but (C) will
break the mesh generator of GSS. The ellipse region is used for
photon crystal simulation. By choosing different division number,
GSS can build triangle, rectangle, hexagon as well as ellipse
(circle). Fig\ref{Region2} shows different shapes of polygons build
by ellipse.
\begin{figure}[ht]
  \hfill
  \begin{minipage}[t]{.5\textwidth}
    \begin{center}
      \includegraphics[scale=0.6]{r1.png}
      \caption{Multi-Region definition.}\label{Region1}
    \end{center}
  \end{minipage}
  \hfill
  \begin{minipage}[t]{.45\textwidth}
    \begin{center}
      \includegraphics[scale=0.3]{region.png}
      \caption{Define shapes of ellipse region}\label{Region2}
    \end{center}
  \end{minipage}
  \hfill
\end{figure}

\newpage
\subsection{SEGMENT}
Segment is a group of boundary edges which have the same attribute.
This statement specifies the label of a special segment. User can
assign the segment with a special boundary type by \textbf{BOUNDARY}
statement.

\subsubsection*{Syntax}
\begin{verbatim}
       SEGMENT Label=<s> { Location=<s> | ( Direction=<s> X=<n> | Y=<n> ) }
               [ {X.MIN=<n> | IX.MIN=<n>} ] [ {X.MAX=<n> | IX.MAX=<n>} ]
               [ {Y.MIN=<n> | IY.MIN=<n>} ] [ {Y.MAX=<n> | IY.MAX=<n>} ]
\end{verbatim}

\small
\noindent\begin{longtable}{ccccp{7cm}}
\textbf{parameter}   & \textbf{type}  & \textbf{default} & \textbf{unit} & \textbf{description} \\
Label       & string  & -       & -                    & Specifies the identifier of this segment, limited to 31 chars. \\
Location    & string  & -       & -                    & Specifies which side the segment lies along. Allowed: TOP, BOTTOM, LEFT or RIGHT.\\
Direction   & string  & -       & -                    & Specifies the dimensional orientation of the segment. Allowed: Horizontal or Vertical. \\
X           & number  & 0.0     & $\mathrm{\mu m}$     & Specifies the X coordinate of the vertical segment. \\
Y           & number  & 0.0     & $\mathrm{\mu m}$     & Specifies the Y coordinate of the horizontal segment. \\
X.MIN       & number  & XMIN    & $\mathrm{\mu m}$     & The minimum x location of the segment.
                                                         synonym: \textbf{X.LEFT}. \\
X.MAX       & number  & XMAX    & $\mathrm{\mu m}$     & The maximum x location of the segment.
                                                         synonym: \textbf{X.RIGHT}. \\
IX.MIN      & integer & 0       & -                    & The minimum x node index of the segment.
                                                         synonym: \textbf{IX.LEFT}.\\
IX.MAX      & integer & IXMAX-1 &-                     & The maximum x node index of the segment.
                                                         synonym: \textbf{IX.RIGHT}. \\
Y.MIN       & number  & YMIN    & $\mathrm{\mu m}$     & The minimum y location of the segment.
                                                         synonym: \textbf{Y.BOTTOM}. \\
Y.MAX       & number  & YMAX    & $\mathrm{\mu m}$     & The maximum y location of the segment.
                                                         synonym: \textbf{Y.TOP}. \\
IY.MIN      & integer & 0       & -                    & The minimum y node index of the segment.
                                                         synonym: \textbf{IY.TOP}. \\
IY.MAX      & integer & IYMAX-1 & -                    & The maximum y node index of the segment.
                                                         synonym: \textbf{IY.BOTTOM}.
\end{longtable}
\normalsize

\subsubsection*{Example}
\begin{verbatim}
            SEGMENT   Label=Anode    Direction=Horizontal X.MIN=0.0 X.MAX=1.0 Y=0.0
            SEGMENT   Label=Cathode  Direction=Horizontal X.MIN=0.0 X.MAX=3.0 Y=-3.0
            SEGMENT   Label=Anode    Location=TOP   X.MIN=0.0 X.MAX=1.0
            SEGMENT   Label=Cathode  Location=BOTTOM
\end{verbatim}

\subsubsection*{Hint}
Here, I have to mention the naming principle of segments. Beside
labeled segments, the interface edges between two regions will be
assigned by IF\_\textit{name1}\_to\_\textit{name2} in which the
\textit{name1} and \textit{name2} is the labels of the two regions
by alpha order. The remain edges of a region will be assigned by
\textit{name}\_Neumann and the \textit{name} is the label of the
region.

One can define a segment for probing data. Please refer to \textbf{PROBE} statement.
This kind of segment should be placed inside a region. Equally, NO intersection to any other segment.

\newpage
\subsection{REFINE}
The \textbf{REFINE} statement allows refinement of a coarse mesh.

\subsubsection*{Syntax}
\begin{verbatim}
       REFINE  Variable=(Doping|Potential) Dispersion=<n> DivisionRatio=<n>
               [ Measure=(Linear|SignedLog) ] [ Triangle=<s> ]
\end{verbatim}

\small \noindent\begin{longtable}{ccccp{7cm}}
\textbf{parameter}   & \textbf{type}  & \textbf{default} & \textbf{unit} & \textbf{description} \\
Variable    & string  & -       & -   & Specifies that the grid refinement is based on the potential or doping quantity. \\
Measure     & string  & Linear  & -   & Specifies that refinement is based on the original value or logarithm of the
                                        specified quantity.\\
Dispersion  & number  & 3.0     & -   & The numerical criterion for refining a triangle. If the specified
                                        quantity differs by more than this parameter at the nodes of a triangle,
                                        the triangle is divided.\\
DivisionRatio & number & 0.25   & -   & The area of divided triangle over area of original triangle.
                                        The default value suggests Triangle code divide one triangle into 4 small triangles.
                                        It is a suggestion value, Triangle code will adjust it for mesh quality reason.\\
Triangle    & string  &praq30Dz & -   & Passes parameters to Triangle code.\\
\end{longtable}
\normalsize

\subsubsection*{Example}
\begin{verbatim}
            REFINE   Variable=Doping Measure=SignedLog Dispersion=1
            REFINE   Variable=Potential Measure=Linear Dispersion=0.1
\end{verbatim}

\begin{figure}[ht]
\centering
\includegraphics[scale=0.3]{refine.png}
\caption{Mesh refinement for a PN diode.}
\end{figure}

\newpage
\section{Doping Profile}
The \textbf{PROFILE} statement defines profiles for impurities to be
used in the device structure. At present, GSS supports analytic
profiles such as uniform, gauss distribution in both x-y directions
and error function distribution in x direction while gauss
distribution in y direction.

\subsubsection*{Syntax}
\begin{verbatim}
       PROFILE { Type=Uniform   |
                 Type=Gauss     [YCHAR=<n> | Y.Junction=<n>] [XCHAR=<n>] |
                 Type=ErrorFunc [YCHAR=<n>] [XCHAR=<n>] }
               Ion=(Donor|Acceptor) { N.Peak=<n> | Dose=<n> }
               [ X.MIN=<n> ] [ X.MAX=<n> ] [ Y.MIN=<n> ] [ Y.MAX=<n> ]
\end{verbatim}

\small
\noindent\begin{longtable}{ccccp{7cm}}
\textbf{parameter}   & \textbf{type}  & \textbf{default} & \textbf{unit} & \textbf{description} \\
Type        & string  & -       & -                    & Specifies that the profile has a uniform, gauss or error function distribution.\\
Ion         & string  & -       & -                    & Specifies the impurity ionization. \\
N.Peak      & number  & 0.0     & $cm^{-3}$            & The peak impurity concentration for an impurity profile.\\
Dose        & number  & 0.0     & $cm^{-2}$            & The dose of the impurity profile assuming a full Gaussian distribution.\\
X.MIN       & number  & 0.0     & $\mathrm{\mu m}$     & The minimum x location of the doping profile.
                                                         synonym: \textbf{X.LEFT}. \\
X.MAX       & number  & XMIN    & $\mathrm{\mu m}$     & The maximum x location of the doping profile.
                                                         synonym: \textbf{X.RIGHT}. \\
Y.MIN       & number  & YMAX    & $\mathrm{\mu m}$     & The minimum y location of the doping profile.
                                                         synonym: \textbf{Y.BOTTOM}. \\
Y.MAX       & number  & 0.0     & $\mathrm{\mu m}$     & The maximum y location of the doping profile.
                                                         synonym: \textbf{Y.TOP}. \\
YCHAR       & number  & 0.25    & $\mathrm{\mu m}$     & The y characteristic length of the profile outside the range of
                                                         \textbf{Y.MIN} < y < \textbf{Y.MAX}. \\
XCHAR       & number  & 0.25    & $\mathrm{\mu m}$     & The x characteristic length of the profile outside the range of
                                                         \textbf{X.MIN} < x < \textbf{X.MAX}. \\
Y.Junction  & number  & 0.0     & $\mathrm{\mu m}$     & The y location under the center of the profile where the magnitude
                                                         of the profile being added equals the magnitude of the
                                                         background profile. \\
\end{longtable}
\normalsize

\subsubsection*{Example}
\begin{verbatim}
        PROFILE   Type=Uniform Ion=Donor     N.PEAK=1E15 \
                  X.MIN=0.0 X.MAX=3.0  Y.TOP=0.0 Y.BOTTOM=-3.0
        PROFILE   Type=Gauss   Ion=Acceptor  N.PEAK=1E18 X.CHAR=0.2 Y.JUNCTION=-0.5 \
                  X.MIN=0.0 X.MAX=0.7  Y.TOP=0.0 Y.BOTTOM=0.0
        PROFILE   Type=ErrorFunc Ion=Acceptor  N.PEAK=2E17 X.CHAR=0.25 Y.CHAR=0.25 \
                  X.MIN=0.5 X.MAX=1.0  Y.TOP=0.0 Y.BOTTOM=0.0
\end{verbatim}


\newpage
\section{Voltage and Current Source}
\subsection{Introduction}
For simulation the transient response of device, GSS supports
several types of voltage and current source. The original models of
these sources come from SPICE, a famous circuit simulation program.
Several sources may be defined in one disk file. And the placement
of these definitions are not critical. The sources can be assigned
to electrode by \textbf{ATTACH} statement when needed.

\subsection{ISOURCE}

\subsubsection*{Syntax}
\begin{verbatim}
        isource Type=IDC  ID=<s> Tdelay=<n> Iconst=<n>
        isource Type=ISIN ID=<s> Tdelay=<n> Iamp=<n> Freq=<n>
        isource Type=IEXP ID=<s> Tdelay=<n> TRC=<n>  TFD=<n>
                TFC=<n> Ilo=<n> Ihi=<n>
        isource Type=IPULSE ID=<s> Tdelay=<n> Tr=<n> Tf=<n>
                Pw=<n> Pr=<n> Ilo=<n> Ihi=<n>
        isource Type=ISHELL ID=<s> DLL=<s> Func=<s>
\end{verbatim}

\small
\noindent\begin{longtable}{ccccp{7cm}}
\textbf{parameter}   & \textbf{type}    & \textbf{default} & \textbf{unit} & \textbf{description} \\
Type   & string  & -  & -    & This parameter declares which type of current source is defined here.
                               Only four types of current source listed as previous are supported at present. \\
ID     & string  & -  & -    & A unique string which identifies the current source.\\
Tdelay & number  & 0  & $\mathrm{s}$  & A proper delay time before the activation of this current source. \\
Iconst & number  & 0  & $\mathrm{mA}$ & The current of the IDC. \\
Iamp   & number  & 0  & $\mathrm{mA}$ & The amplitude current of the ISIN. \\
Freq   & number  & 0  & $\mathrm{Hz}$ & The frequency of the ISIN. \\
TRC    & number  & 0  & $\mathrm{s}$  & The rise time constant of the IEXP. \\
TFD    & number  & 0  & $\mathrm{s}$  & The fall delay time of the IEXP. \\
TFC    & number  & 0  & $\mathrm{s}$  & The fall time constant of the IEXP. \\
Tr     & number  & 0  & $\mathrm{s}$  & The raise edge of the IPULSE. \\
Tf     & number  & 0  & $\mathrm{s}$  & The fall edge of the IPULSE. \\
Pw     & number  & 0  & $\mathrm{s}$  & The pulse with of the IPULSE. \\
Pr     & number  & 0  & $\mathrm{s}$  & The period of the IPULSE. \\
Ilo    & number  & 0  & $\mathrm{mA}$ & The low current for both IEXP and IPULSE. \\
Ihi    & number  & 0  & $\mathrm{mA}$ & The high current for both IEXP and IPULSE. \\
DLL    & string  & -  & -             & The name of dynamic library file.\\
Func   & string  & -  & -             & The name of the function loaded from dynamic library file.
\end{longtable}
\normalsize

\subsubsection*{Example}
\begin{verbatim}
        isource Type=IDC    ID=I1 Tdelay=0 Iconst=5
        isource Type=ISIN   ID=I2 Tdelay=0 Iamp=0.1 Freq=1e6
        isource Type=IEXP   ID=I3 Tdelay=0 TRC=1E-6 TFD=3E-6 TFC=1E-6 Ilo=0 Ihi=1
        isource Type=IPULSE ID=I4 Tdelay=0 Tr=1E-9 Tf=1E-9 Pw=5E-6 Pr=1E-5 Ilo=0 Ihi=1
\end{verbatim}

\newpage
\subsection{VSOURCE}
\subsubsection*{Syntax}

\begin{verbatim}
        vsource Type=VDC  ID=<s> Tdelay=<n> Vconst=<n>
        vsource Type=VSIN ID=<s> Tdelay=<n> Vconst=<n> Vamp=<n> Freq=<n> Alpha=<n>
        vsource Type=VEXP ID=<s> Tdelay=<n> TRC=<n>  TFD=<n>
                TFC=<n> Vlo=<n> Vhi=<n>
        vsource Type=VPULSE ID=<s> Tdelay=<n> Tr=<n> Tf=<n>
                Pw=<n> Pr=<n> Vlo=<n> Vhi=<n>
        vsource Type=VSHELL ID=<s> DLL=<s> Func=<s>
\end{verbatim}

\small
\noindent\begin{longtable}{ccccp{7cm}}
\textbf{parameter}   & \textbf{type}  & \textbf{default} & \textbf{unit} & \textbf{description} \\
Type   & string  & -  & -             & This parameter declares which type of
                                        voltage source is defined here.
                                        Only four types of voltage source listed as previous are supported at present. \\
ID     & string  & -  & -             & A unique string which identifies the voltage source.\\
Tdelay & number  & 0  & $\mathrm{s}$  & A proper delay time before the activation of this voltage source. \\
Vconst & number  & 0  & $\mathrm{V}$  & The voltage of the VDC. \\
Vamp   & number  & 0  & $\mathrm{V}$  & The amplitude voltage of the VSIN. \\
Freq   & number  & 0  & $\mathrm{Hz}$ & The frequency of the VSIN. \\
Alpha  & number  & 0  & -             & The exponential attenuation parameter of the VSIN. \\
TRC    & number  & 0  & $\mathrm{s}$  & The rise time constant of the VEXP. \\
TFD    & number  & 0  & $\mathrm{s}$  & The fall delay time of the VEXP. \\
TFC    & number  & 0  & $\mathrm{s}$  & The fall time constant of the VEXP. \\
Tr     & number  & 0  & $\mathrm{s}$  & The raise edge of the VPULSE. \\
Tf     & number  & 0  & $\mathrm{s}$  & The fall edge of the VPULSE. \\
Pw     & number  & 0  & $\mathrm{s}$  & The pulse with of the VPULSE. \\
Pr     & number  & 0  & $\mathrm{s}$  & The period of the VPULSE. \\
Vlo    & number  & 0  & $\mathrm{V}$  & The low voltage for both VEXP and VPULSE. \\
Vhi    & number  & 0  & $\mathrm{V}$  & The high voltage for both VEXP and VPULSE. \\
DLL    & string  & -  & -             & The name of dynamic library file.\\
Func   & string  & -  & -             & The name of the function loaded from dynamic library file.
\end{longtable}
\normalsize

\subsubsection*{Example}
\begin{verbatim}
        vsource Type=VDC    ID=GND  Tdelay=0 Vconst=0
        vsource Type=VDC    ID=VCC  Tdelay=0 Vconst=5
        vsource Type=VSIN   ID=Vs   Tdelay=1e-6 Vamp=0.1 Freq=1e6
        vsource Type=VEXP   ID=V1   Tdelay=0 TRC=1e-6 TFD=1e-6 TFC=1e-6 Vlo=0 Vhi=1
        vsource Type=VPULSE ID=V2   Tdelay=0 Tr=1e-9 Tf=1e-9 Pw=5e-6 Pr=1e-5 Vlo=0 Vhi=1
        vsource Type=VSHELL ID=VGauss  DLL=foo.so Func=vsrc_gauss
\end{verbatim}

\newpage
\subsubsection*{Hint}
GSS supports user defined voltage and current source by loading
shared object (.so) file. The file which contains a user defined
voltage source should have the function as follow. GSS will pass the
argument time in the unit of second to the function
\textit{vsrc\_name} and get voltage value in the unit of volt. The
current source function is almost the same except the unit of
current is $\mathrm{mA}$.

\begin{verbatim}
    double vsrc_name(double time)
    {
       /* calculate the voltage amplitude */
       return vsrc_amplitude;
    }
    double isrc_name(double time)
    {
       /* calculate the current amplitude */
       return isrc_amplitude;
    }
\end{verbatim}

The c code should be linked with -shared and -fPIC option as:
\begin{verbatim}
    gcc -shared -fPIC -o foo.so foo.c -lm
\end{verbatim}

The \textit{foo.so} file should be put in the same directory as
input file.

\newpage
\section{Boundary Condition}

\subsection{BOUNDARY and CONTACT}
The \textbf{BOUNDARY} statement sets boundary information to
representing segments which defined by mesh generator or read from
CGNS file.

GSS now fully support electrode region (the material of this region
may be metal or poly-Si). One should use \textbf{CONTACT} statement
to specify the electrode type of this region(s).

\subsubsection*{Syntax}
\begin{verbatim}
        BOUNDARY Type=OhmicContact       ID=<s>  [ Res=<n> ] [ Cap=<n> ] [ Ind=<n> ]
                 [ Heat.Transfer=<n> ] [EXT.Temp=<n> ] [ConnectTo=<s>]
        BOUNDARY Type=SchottkyContact    ID=<s>  [ Res=<n> ] [ Cap=<n> ] [ Ind=<n> ]
                 WorkFunction=<n> [ Heat.Transfer=<n> ] [EXT.Temp=<n> ]
        BOUNDARY Type=GateContact        ID=<s>  WorkFunction=<n>
                 [ Res=<n> ] [ Cap=<n> ] [ Ind=<n> ]
                 [ Heat.Transfer=<n> ] [EXT.Temp=<n> ]
        BOUNDARY Type=InsulatorContact   ID=<s>  WorkFunction=<n> [ QF=<n> ]
                 [ Res=<n> ] [ Cap=<n> ] [ Ind=<n> ]
                 Thickness=<n> Eps=<n>  [ Heat.Transfer=<n> ] [EXT.Temp=<n> ]
        BOUNDARY Type=InsulatorInterface ID=<s>  [ QF=<n> ]
        BOUNDARY Type=Heterojunction     ID=<s>  [ QF=<n> ]
        BOUNDARY Type=NeumannBoundary    ID=<s> [ Heat.Transfer=<n> ] [EXT.Temp=<n> ]

        CONTACT  Type=OhmicContact       ID=<s>
                 [ Res=<n> ] [ Cap=<n> ] [ Ind=<n> ] [ConnectTo=<s>]
                 [ Heat.Transfer=<n> ] [EXT.Temp=<n> ]
        CONTACT  Type=SchottkyContact    ID=<s>
                 [ Res=<n> ] [ Cap=<n> ] [ Ind=<n> ]
                 WorkFunction=<n> [ Heat.Transfer=<n> ] [EXT.Temp=<n> ]
        CONTACT  Type=GateContact        ID=<s>  WorkFunction=<n>
                 [ Res=<n> ] [ Cap=<n> ] [ Ind=<n> ]
                 [ Heat.Transfer=<n> ] [EXT.Temp=<n> ]
        CONTACT  Type=FloatMetal         ID=<s>  [ QF=<n> ]
\end{verbatim}

\small
\noindent\begin{longtable}{ccccp{7cm}}
\textbf{parameter}   & \textbf{type}    & \textbf{default} & \textbf{unit} & \textbf{description} \\
Type          & string  & -  & -    & This parameter declares which type of boundary condition is defined here.\\
ID            & string  & -  & -    & A unique string which identifies the corresponding segment.\\
Res           & number  & 0  & $\Omega$      & The lumped resistance for the electrode.  \\
Cap           & number  & 0  & $\mathrm{F}$  & The lumped capacitance for the electrode. \\
Ind           & number  & 0  & $\mathrm{H}$  & The lumped inductance for the electrode. \\
ConnectTo     & string  & -  & -             & Specifies the ID of an ohmic electrode which connect to this ohmic electrode. Useful for CMOS structure.\\
WorkFunction  & number  &4.7 & $\mathrm{V}$  & The workfunction of the Schottky contact or gate material. \\
QF            & number  & 0  & $\mathrm{C}\cdot \mathrm{cm}^{-2}$  & For InsulatorContact and InsulatorInterface bc: The surface charge density of semiconductor-insulator interface. \\
QF            & number  & 0  & $\mathrm{C}\cdot \mathrm{cm}^{-2}$  & For Heterojunction bc: The surface charge density of heterojunction. \\
QF            & number  & 0  & $\mathrm{C}\cdot \mathrm{\mu m}^{-1}$  & For FloatMetal bc: The free charge per micron in Z dimension. \\
Thickness     & number  &2e-7  & $\mathrm{cm}$ & The thickness of $\mathrm{SiO}_2$ layer. \\
Eps           & number  &3.9   & -             & The relative permittivity of $\mathrm{SiO}_2$ layer.\\
Heat.Transfer & number  &1e3   &$\mathrm{W}/(\mathrm{cm}\cdot \mathrm{K})$   & The heat transfer rate of boundary. \\
EXT.Temp      & number  & LatticeTemp  & $\mathrm{K}$   & The external temperature. \\
\end{longtable}
\normalsize

\subsubsection*{Example}
\begin{verbatim}
        BOUNDARY Type=InsulatorContract  ID=SiSiO2    Res=0 Cap=0 Ind=0 \
                 Thickness=1e-6 Eps=3.9 WorkFunction=4.7 QF=0
        BOUNDARY Type=InsulatorInterface ID=IFACE     QF=0
        BOUNDARY Type=GateContract       ID=GATE      Res=0 Cap=0 Ind=0 WorkFunction=4.7
        BOUNDARY Type=NeumannBoundary    ID=WALL      Heat.Transfer=0 EXT.Temp=300
        BOUNDARY Type=SchottkyContract   ID=sgate     Res=0 Cap=0 Ind=0 VBarrier=0.8
        BOUNDARY Type=OhmicContract      ID=OMANODE   Res=0 Cap=0 Ind=0
        BOUNDARY Type=OhmicContract      ID=OMCATHODE Res=0 Cap=0 Ind=0
\end{verbatim}

\subsubsection*{Hint}
Four "electrode" boundary conditions are supported by GSS. The names
are ended with "Contact". The OhmicContact and SchottkyContact
electrodes have current flow in both steady state and transient
situations. While GateContact and InsulatorContact(a simplified
MOSFET Gate boundary condition) only have displacement current in
transient situation.

GSS supports five interfaces which can be set automatically:
semiconductor-insulator interface(InsulatorInterface),
semiconductor-electrode interface(set to OhmicContract as default),
interface between different semiconductor material(Heterojunction)
and  interface between same semiconductor material(Homojunction).
These boundaries can be set automatically by GSS if user didn't set
them explicitly. However, the electrode-insulator interface, may
have several situations: Gate to Oxide interface, FloatMetal to
Oxide interface or Source/Drain electrode to Oxide interface. As a
result, this interface can only be set correctly when electrode type
is known. Please refer to the following \textbf{CONTACT} statement.

GSS can build region with metal or poly-Si material to form an
electrode. Which means, i.e. for OhmicContact bc, one can simply
specify a segment as Ohmic bc or build an electrode region as Ohmic
electrode. Since Version 0.45.03, GSS considers electrode region,
semiconductor region and insulator region during calculation. As a
result, GSS added \textbf{CONTACT} statement for fast boundaries
specification of electrode region. At present, GSS support electrode
with the type of Ohmic, Schottky, Gate and FloatMetal. All the
electrode should be specified explicitly and GSS will set
corresponding boundaries automatically.

The "ID" parameter of \textbf{BOUNDARY} statement is limited to
segment label. And The "ID" parameter of \textbf{CONTACT} statement
is limited to region name.

The NeumannBoundary, which is the default boundary type for all the
non-interface segments, can also be set automatically.

\newpage
\subsection{ATTACH}
This statement is used to add voltage or current sources to the
electrode boundary. The statement first clears all the sources
connected to the specified electrode and then adds source(s) defined
by VApp or IApp parameter. If two or more sources are attached to
the same electrode, the total effect is the summation of all
sources. However, the sources attached to one electrode must have
the same type.

\subsubsection*{Syntax}
\begin{verbatim}
        ATTACH Electrode=<s>  Type=Voltage VApp=<s> [VApp=<s> ...]
        ATTACH Electrode=<s>  Type=Current IApp=<s> [IApp=<s> ...]
\end{verbatim}

\small \noindent\begin{longtable}{ccccp{7cm}}
\textbf{parameter}   & \textbf{type}    & \textbf{default} & \textbf{unit} & \textbf{description} \\
Electrode     & string  & -  & -    & Specifies which electrode boundary is to be attached with one or more sources.\\
Type          & string  & Voltage  & -    & The sources are voltage or current type.\\
VApp          & string  & -  & -    & Specifies the ID of voltage source which is to be attached to this electrode.\\
IApp          & string  & -  & -    & Specifies the ID of current source which is to be attached to this electrode.\\
\end{longtable}
\normalsize

\subsubsection*{Example}
\begin{verbatim}
        ATTACH   Electrode=Collector VApp=VCC
        ATTACH   Electrode=Emitter   VApp=GND
        ATTACH   Electrode=Base      VApp=Vb  VApp=Vs
        ATTACH   Electrode=Base      Type=Current  IApp=Ib  IApp=Is
\end{verbatim}

\subsubsection*{Hint}
If electrode is attached with voltage source(s), the R, C and L
defined by \textbf{BOUNDARY} statement will affect later simulation.
But solver will ignore those lumped elements with the electrode
which stimulated by current source(s). Please refer to Fig
\ref{Electrode}.

The positive direction of current is flow into the electrode.

Only Ohmic and Schottky electrodes can be attached by current
source(s).

If no source attached explicitly, the electrode is set to be
attached to ground.

\begin{figure}[ht]
\centering
\includegraphics[scale=0.4]{electrode.png}
\caption{Voltage and current boundary.} \label{Electrode}
\end{figure}


\newpage
\section{Physical Model Interface}
GSS use a dynamic mechanician to support various materials and
physical models. Each material has a dynamic load library (.so)
which contains its physical parameters. User can modify the
parameters which can be found at \$(GSS\_DIR)/src/material and
recompile it. Experts can even offer their own physical model files.

At present, GSS has a \textbf{PMIS} statement for choosing different
mobility models and impact ionization models.

\subsubsection*{Syntax}
\begin{verbatim}
        PMIS Region=<s>  Mobility=<s>  II.Model=<s>
\end{verbatim}

\small \noindent\begin{longtable}{ccccp{7cm}}
\textbf{parameter}   & \textbf{type}    & \textbf{default} & \textbf{unit} & \textbf{description} \\
Region        & string  & -         & -    & Specifies the semiconductor region which use the following physical model.\\
Mobility      & string  & Analytic  & -    & The mobility model name.\\
II.Model      & string  & Default   & -    & The impact ionization model name.\\
\end{longtable}
\normalsize

\subsubsection*{Example}
\begin{verbatim}
        PMIS Region=Si  Mobility=Philips
        PMIS Region=Si  Mobility=Lucent  II.Model=Valdinoci
\end{verbatim}

\subsubsection*{Hint}
One can set different physical models to individual region.

GSS has implemented Analytic, Philips and Lucent mobility model for
all the supported material. The Analytic and Philips mobility model
only takes parallel field effect and they can be used within all the
four solvers. The author suggest to use these models for bipolar
device simulations. The Lucent mobility model, which considers
parallel and transverse electrical field, is an accurate model for
MOS structure. But it should work with DDML1E/DDML2E solvers in
which transverse electrical field is calculated. The Lombardi and HP
(Hewlett-Packard) mobility model only validate for Silicon. These
two mobility models include parallel and transverse electrical field
corrections and can be used for MOSFET simulation. The Hypertang
mobility model only validate for GaAs. It is reported that this
model can avoid unrealistic drain current oscillation when applied
to the simulation of GaAs MESFET.

The impact ionization model is still very limited in GSS. Only
Valdinoci model for silicon is valid at present.

\newpage
\section{Solve Specification}
\subsection{Introduction}
These statements instruct GSS core to perform user specified solution(s).


\subsection{METHOD}
The \textbf{METHOD} statement sets the solver and the parameters of
the solver. At present, GSS 0.4x has basic DDM solver(DDML1E),
lattice temperature corrected DDM solver(DDML2E) and EBML3E solver which base on
energy balance model.

\subsubsection*{Syntax}
\begin{verbatim}
        METHOD Type=(DDML1E|DDMLE2|EBML3E|QDDML1E)  Scheme=Newton
               HighFieldMobility=(On|Off)   EJModel=(On|Off)
               ImpactIonization=(On|Off)    II.Type=(EdotJ|EVector|ESide|GradQf)
               BandBandTunneling=(On|Off)
               Fermi=(On|Off)
               NS=(Basic|LineSearch|TrustRegion)
               LS=(SuperLU|LU|CGS|BICG|BCGS|GMRES|TFQMR)
               Damping=(BankRose|Potential|No)
               MaxIteration=<i> relative.tol=<n>
               possion.tol=<n>  elec.continuty.tol=<n>  hole.continuty.tol=<n>
               elec.energy.tol=<n> hole.energy.tol=<n>  latt.temp.tol=<n>
               electrode.tol=<n> toler.relax=<n>
               QNFactor=<n> QPFactor=<n>
\end{verbatim}

\small \noindent\begin{longtable}{ccccp{7cm}}
\textbf{parameter}   & \textbf{type}    & \textbf{default} & \textbf{unit} & \textbf{description} \\
Type              & string  & DDML1       & -    & Specifies the solver.\\
Scheme            & string  & Newton      & -    & At present, GSS only supports Newton's full iterative scheme.\\
HighFieldMobility & bool    & On          & -    & Specifies if high field mobility should be used. GSS set this flag to OFF for equilibrium state.\\
EJModel           & bool    & Off         & -    & Specifies if EdotJ and EcrossJ should be use to calculate high field mobility.
                                                   GSS will use a simpler model when this flag is set to OFF.\\
ImpactIonization  & bool    & Off         & -    & Specifies if impact ionization should be considered.\\
II.Type           & string  & GradQf      & -    & Specifies the implement model of impact ionization. \\
BandBandTunneling & string  & Off         & -    & Specifies if band to band tunneling should be considered.\\
Fermi             & bool    & Off         & -    & Specifies if Fermi-Dirac statistics should be considered.\\
NS                & string  &LineSearch   & -    & Specifies the nonlinear solver. \\
LS                & string  &GMRES        & -    & Specifies the linear solver. \\
Damping           & string  & No          & -    & Load a Newton damping method for LineSearch or Basic Newton nonlinear solver.\\
MaxIteration      & integer & 30          & -    & The max number of iteration nonlinear solver will try. But for equilibrium state calculation, the max
                                                    allowed iteration number is 10 times more than this value.\\
relative.tol      & number  & 1e-5        & -    & When relative error of solution variable less than this value, solution is considered converged.\\
possion.tol       & number  & 1e-26       &$\rm{C\cdot {\mu m}^{-1}} $ & The absolute converged criteria for the Poisson equation.\\
elec.continuty.tol& number  & 5e-18       &$\rm{A\cdot {\mu m}^{-1}} $ & The absolute converged criteria for the electron continuity equation.\\
hole.continuty.tol& number  & 5e-18       &$\rm{A\cdot {\mu m}^{-1}} $ & The absolute converged criteria for the hole continuity equation.\\
elec.energy.tol   & number  & 1e-18       &$\rm{W\cdot {\mu m}^{-1}} $ & The absolute converged criteria for the electron energy balance equation.\\
hole.energy.tol   & number  & 1e-18       &$\rm{W\cdot {\mu m}^{-1}} $ & The absolute converged criteria for the hole energy balance equation.\\
latt.temp.tol     & number  & 1e-11       &$\rm{W\cdot {\mu m}^{-1}} $ & The absolute converged criteria for the lattice heat equation equation.\\
electrode.tol     & number  & 1e-9        &$\rm{V}$                    & The absolute converged criteria for the electrode bias equation.\\
toler.relax       & number  & 1e4         & -                          & When relative error is used as converged criteria, the equation norm should
                                                                         satisfy the absolute converged criteria with a relaxation of this value.\\
QNFactor          & number  & 1.0         & -                          & The damping quantity of electron quantum potential.\\
QPFactor          & number  & 1.0         & -                          & The damping quantity of hole quantum potential.\\
\end{longtable}
\normalsize

\subsubsection*{Example}
\begin{verbatim}
        METHOD   Type=DDML1E   Scheme=Newton  NS=LineSearch  LS=GMRES
        METHOD   Type=DDML1E   Scheme=Newton  NS=TrustRegion LS=LU
        METHOD   Type=DDML2E   Scheme=Newton  NS=Basic LS=TFGMR Damping=Potential
\end{verbatim}

\subsubsection*{Hint}
All the DDML1E/DDML2E/EBML3E/QDDML1E solvers support parallel and
transverse electrical field dependent mobility.

Lattice temperature equation is considered by DDML2E solver. The
EBML3E solver is based on advanced energy balance method. The
QDDML1E is a density-gradient solver which consists of quantum
correction to classical model.

The carrier generation by impact ionization and band band tunneling
is really difficult for calculation. However, DDML1E/DDML2E solvers
are carefully designed for impact ionization and band band tunneling
calculation, i.e. diode reverse breakdown simulation. Usually, the
temperature can't keep unchanged if carrier generation takes place.
As a result, DDML2E solver is highly recommend for these types of
situations. At present, EBML3E and QDDML1E solver don't support
impact ionization.

Fermi statistics is only supported by DDML1E and DDML2E solvers.

LineSearch and TrustRegion accelerating methods work well when
initial value a bit far from real solution, e.g. first time
computing. Basic Newton method should only be used when initial
value is near the true solution, e.g. dc sweep and transient
calculation.

Each nonlinear solver should have a inner linear solver. To choose a
suitable linear solver may help the convergence. The performance of
LineSearch and Basic Newton methods is good when Krylov subspace
linear solvers(CGS, BICG, BCGS, GMRES and TFQMR) are employed.
However, the TrustRegion method prefers LU factorization linear
solver to Krylov subspace linear solvers.

Newton Damping is a useful tool for helping convergence,
especially for the Basic Newton method.

QNFactor and QPFactor is used to enforce the convergence property of QDDML1E solver.
Since quantum solution differs much from classical solution near Si/SiO2 interface, setting
these two factors with small value i.e. 1e-4 and varying it gradually to 1.0, with each
step the solution can get convergence. At last, the value of QXFactor of 1.0
means that the quantum model is fully turned on and applied.

The parameters of \textbf{METHOD} statement will not be affected by previous \textbf{METHOD} statement.

The convergence is considered to be achieved when either the X norm
or the function residual norm falls below certain tolerance. When
every function's residual norm falls small than certain tolerance,
the absolute convergence is achieved. For X norm criteria, it should
fall below \textbf{relative.tol} and every function residual norm
should fit the relaxed (with the relaxation value of
\textbf{toler.relax)} absolute converged criteria.

\newpage
\subsection{SOLVE}
The \textbf{SOLVE} statement instructs GSS to perform a solution for one or more
specified bias points.

\subsubsection*{Syntax}
\begin{verbatim}
        SOLVE Type=EQUILIBRIUM
        SOLVE Type=STEADYSTATE
        SOLVE Type=DCSWEEP    VScan=<s>  [VScan=<s> ...]  [IVRecord=<s> ...]
              [IVFile=<s>]    VStart=<s> VStep=<s> VStop=<n>
        SOLVE Type=DCSWEEP    IScan=<s>  [IVRecord=<s> ...]
              [IVFile=<s>]    IStart=<s> IStep=<s> IStop=<n>
        SOLVE Type=TRANSIENT  ODE.Formula=(BDF1|BDF2) [IVRecord=<s> ...]
              [IVFile=<s>]    TStart=<n> TStep=<n> TStop=<n>
              AutoStep=<b>    Predict=<b>
\end{verbatim}

\small \noindent\begin{longtable}{ccccp{7cm}}
\textbf{parameter}   & \textbf{type}    & \textbf{default} & \textbf{unit} & \textbf{description} \\
Type          & string  & -  & -    & Specifies the Solve condition.\\
VScan         & string  & -  & -    & Specifies the voltage variational electrode boundary for DCSWEEP.\\
VStart        & number  & -  & $\mathrm{V}$    & The initial voltage for DC sweep.\\
VStep         & number  & -  & $\mathrm{V}$    & The voltage step size of DC sweep.\\
VStop         & number  & -  & $\mathrm{V}$    & The finish voltage for DC sweep.\\
IScan         & string  & -      & -    & Specifies the current variational electrode boundary for DCSWEEP.\\
IStart        & number  & -  & $\mathrm{mA}$    & The initial current for DC sweep.\\
IStep         & number  & -  & $\mathrm{mA}$    & The current step size of DC sweep.\\
IStop         & number  & -  & $\mathrm{mA}$    & The finish current for DC sweep.\\
IV.Record     & string  & -      & -    & Specifies which electrode's IV data should be recorded.
                                          User can define serval electrodes here. \\
IV.File       & string  & -      & -    & Specifies the file which contains the IV data.\\
ODE.Formula   & string  & BDF2   & -    & Specifies the time march scheme for solving the time-domain ordinary differential equation.\\
TStart        & number  & -  & $\mathrm{s}$    & The initial time for transient calculation.\\
TStep         & number  & -  & $\mathrm{s}$    & The time step size of transient calculation.\\
TStop         & number  & -  & $\mathrm{s}$    & The finish time for transient calculation.\\
AutoStep      & bool    & -  & True            & Use automatically time step control based on LTE.\\
Predict       & bool    & -  & True            & Predict initial value for next time step.\\
\end{longtable}
\normalsize

\subsubsection*{Example}
\begin{verbatim}
        SOLVE    Type=EQUILIBRIUM
        SOLVE    Type=DCSWEEP     VScan=Anode    IVRecord=Anode IVRecord=Cathode \
                 IVFile=ivfp.txt  VStart=0 VStep=1e-2 VStop=0.6
        SOLVE    Type=DCSWEEP     IScan=Anode    IVRecord=Anode IVRecord=Cathode \
                 IVFile=ivfp2.txt IStart=0.02 IStep=1e-2 IStop=1
        SOLVE    Type=TRANSIENT  IVRecord=Anode  IVFile=iv.txt \
                 TStart=0 TStep=1e-10  TStop=3e-8
\end{verbatim}

\subsubsection*{Hint}

For equilibrium state calculation, all the electrodes are set to
ground.

You can't do a DC sweep with current scan to GateContact and
InsulatorContact.

When STEADYSTATE or DCSWEEP solve is performed, transient 0 value of
the voltage(current) source will be used as the bias of each
electrode.

One can do voltage DCSWEEP with multi-electrode by specifying two or more VScan parameter.
The voltage will be assigned to each electrode during the simulation. This function is useful
for Double Gate MOS simulation.

The step size for DCSWEEP calculation will
automatically reduce to half size if last step diverged. Then it
will be multiplied by 1.1 on each step until it reaches original
step size.

TRANSIENT simulation now use automatically time step control based on LTE (local truncation error).

\newpage
\subsection{AC Sweep Solver}
In addition to DC steady state and transient analysis, GSS now
allows AC small-signal analysis as a post-processing step after a DC
solution.

\subsubsection*{Syntax}
\begin{verbatim}
        METHOD Type=DDML1AC   LS=(LU|CGS|BICG|BCGS|GMRES|TFQMR)
               HighFieldMobility=(On|Off)   EJModel=(On|Off)
               ImpactIonization=(On|Off)    II.Type=(EdotJ|EVector|ESide|GradQf)
               BandBandTunneling=(On|Off)
               Fermi=(On|Off)
        SOLVE  Type=ACSWEEP    ACScan=<s>  [IVRecord=<s> ...]
               [IVFile=<s>]    FStart=<s> FMultiple=<s> FStop=<n>   VAC=<n>
\end{verbatim}

\small \noindent\begin{longtable}{ccccp{7cm}}
\textbf{parameter}   & \textbf{type}    & \textbf{default} & \textbf{unit} & \textbf{description} \\
ACScan        & string  & -      & -    & Specifies the electrode for ACSWEEP.\\
FStart        & number  & 1e6   & $\mathrm{Hz}$    & The initial frequency for AC sweep.\\
FMultiple     & number  & 1.1   & -                & The multiplicative factor for incrementing frequency.\\
FStop         & number  & 1e9   & $\mathrm{Hz}$    & The finish frequency for AC sweep.\\
VAC           & number  & 0.0026  & $\mathrm{V}$  & The magnitude of the applied small-signal bias.\\
\end{longtable}
\normalsize

\subsubsection*{Hint}
This solver shared Jacobian Matrix with DDML1E solver. Which means
one should call it directly after DDML1E, keeping all the parameters
unchanged for \textbf{METHOD} statement. If a previous computed
result is imported, call DDML1E to do a steady-state calculation
again and run DDML1AC later.

The convergence may be difficult if frequency is very high, i.e.
nearly cut off frequency, because of the poor condition number of
Jacobian matrix.

\newpage
\subsection{EM FEM Solver}
GSS has a electromagnetic solver based on finite element method.
This solver calculates the distribution of electromagnetic field
radiated by monochrome (light) wave. The photon generated carrier
density in semiconductor region can be got at the same time.

\subsubsection*{Syntax}
\begin{verbatim}
        PHOTOGEN  WAVELEN=<n> INTENSITY=<n>  [ANGLE=<n>]  WTM=<n> WTE=<n>
                  [phase.diff=<n>] [quan.eff=<n>]
        METHOD    Type=EMFEM     [LS=LU]
        SOLVE
        LSOURCE   Type=UNIFORM   Tdelay=<n> Power=<n>
        LSOURCE   Type=PULSE     Tdelay=<n> Tr=<n> Tf=<n> Pw=<n> Pr=<n>
                  Powerhi=<n>    Powerlo=<n>
        LSOURCE   Type=LSHELL    DLL=<s> Func=<s>
\end{verbatim}

\textbf{Syntax for PHOTOGEN}
\small \noindent\begin{longtable}{ccccp{8cm}}
\textbf{ parameter}   & \textbf{type}         & \textbf{default} & \textbf{unit} & \textbf{description} \\
WAVELEN       & number  & 0.532 & $\mathrm{\mu m}$              & The wavelength of incident monochrome wave.\\
INTENSITY     & number  & 1.0   & $\mathrm{W \cdot cm^{-2}}$    & The power density of incident wave.\\
ANGLE         & number  & 90    & degree                        & The clockwise angle of the ray direction relative to the horizontal axis.\\
WTM           & number  & 1.0   & -                             & The percentage of intensity of TM model.\\
WTE           & number  & 0.0   & -                             & The percentage of intensity of TE model.\\
phase.diff    & number  & 0.0   & degree                        & The differentiation of phase angle between TE model and TM model.
                                                                  $\Delta \Phi=\Phi_{TM}-\Phi_{TE}$\\
quan.eff      & number  & 1.0   & -                             & The quantum efficiency (which means electron-hole pares generated by one photon)
                                                                  of photon generation.\\
\end{longtable}
\normalsize

\textbf{Syntax for LSOURCE}
\small \noindent\begin{longtable}{ccccp{8cm}}
\textbf{ parameter}   & \textbf{type}         & \textbf{default} & \textbf{unit} & \textbf{description} \\
Type   & string  & -     & -                             & The type of light source.\\
Tdelay & number  & 0.0   & $\mathrm{s}$                  & The delay time before the activation of the light source. \\
Tr     & number  & 1e-15 & $\mathrm{s}$                  & The rise time of the intensity of the pulse-type light source. \\
Tf     & number  & 1e-15 & $\mathrm{s}$                  & The fall time of the intensity of the pulse type light source. \\
Pw     & number  & 0     & $\mathrm{s}$                  & The pulse width of the intensity of the pulse type light source. \\
Pr     & number  & 0     & $\mathrm{s}$                  & The repetition period of the intensity of the pulse type light source. \\
Power  & number  & 1.0   & -                             & The multiply factor to photon generated carrier density.\\
Powerhi& number  & 1.0   & -                             & The higher multiply factor to photon generated carrier density. \\
Powerlo& number  & 0     & -                             & The lower multiply factor to photon generated carrier density. \\
DLL    & string  & -  & -                                & The name of dynamic library file.\\
Func   & string  & -  & -                                & The name of the function loaded from dynamic library file which calculates power coefficient.\\
\end{longtable}
\normalsize

\subsubsection*{Hint}
User need to build a vacuum region surrounding device and a PML
region surrounding vacuum region. These two region should have a
thickness of no less than one wave length.

The work flow of EMFEM solver shows as follows. GSS set its internal
solver to EMFEM when meets \textbf{METHOD} command with
\textbf{EMFEM} type. The actual solving action takes place when
meets the next \textbf{SOLVE} command. GSS will search the first
\textbf{PHOTOGEN} command in the input list, using the parameters in
this command during the solve procedure. This \textbf{PHOTOGEN}
command will be removed from input list after solving action. As a
result, user can set multi \textbf{PHOTOGEN} statements and repeat
\textbf{SOLVE} command for corresponding times to calculate several
beams of monochrome wave, during which the photon generated carrier
density will be added to previous result.

The iterative method such as GMRES usually leads to divergence when
solving FEM problem. LU factorization is highly recommend.

EMFEM only gets the photon generated carrier density. User should
set \textbf{one LSOURCE} to describe the time evolution of the light
source. The actual photon generated carrier density used in
semiconductor simulation is the original value multiplied with
\textbf{power} coefficient specified within \textbf{LSOURCE}.

When DDML1E or DDML2E solver is loaded for further simulation, the
photon generated carrier will be considered.

User can define their own light source by dynamic loaded library as
voltage or current source. Here is a template.
\begin{verbatim}
foo.c:
    double lsrc_power(double time) /* in the unit of s */
    {
       double power;
       /* calculate the power of light source */
       return power;
    }
\end{verbatim}


\newpage
\subsection{IV File Format}
GSS can generate IV record file for DC sweep, transient and AC sweep
calculations. Here is the file format for the three situations.

The file for DC sweep: The first line is begin with '\#', followed
by the name of each electrode. The remain part is the potential and
current for each electrode, each takes one column. The unit of
potential is volt and the unit of current is mA.

The file for transient calculation is nearly the same as the file
for DC sweep, besides that the first column is the time with the
unit of ps.

The file for AC sweep has the same head as above. The remaining part
is organized as follows: The first column is the frequency with the
unit of MHz. Then the IV properties of each electrode. Each
electrode takes six columns, the real, image and amplitude of
potential, followed by three columns for current.

Note: the electrode potential may not equal to the application
voltage if lumped elements take place.

\newpage
\section{File I/O}
\subsection{Introduction}
The \textbf{IMPORT} and \textbf{EXPORT} statements are used to read
and write solutions from a CGNS or TIF file. A model CGNS file only
contains semiconductor device structure while a core CGNS file has
previous solution data besides device structure. The TIF(Technology
Interchange Format) file is an ASCII file used by Synopsys Medici
software which equivalence to core CGNS file. We offer a small code
TIFTool which can open TIF file, view the mesh and solution data and
convert it to CGNS file.

\subsection{IMPORT and EXPORT}
\subsubsection*{Syntax}
\begin{verbatim}
        IMPORT  CoreFile=<s> | ModelFile=<s>
        EXPORT  CoreFile=<s>   [ AscFile=<s> ]  [ VTKFile=<s> ]
\end{verbatim}

\small \noindent\begin{longtable}{ccccp{7cm}}
\textbf{parameter}   & \textbf{type}    & \textbf{default} & \textbf{unit} & \textbf{description} \\
CoreFile      & string  & -  & -    & Write/read device structure and solution data to a CGNS file.\\
ModelFile     & string  & -  & -    & Read device structure from a CGNS file which probably crated by SGframework
                                      or converted from Medici TIF file by TIFTool.\\
AscFile       & string  & -  & -    & Write device structure and solution data to a TIF file.
                                      At present, we can't make our TIF file be accepted by Medici.\\
VTKFile       & string  & -  & -    & Write mesh and solution data to VTK file.\\
\end{longtable}
\normalsize

\subsubsection*{Example}
\begin{verbatim}
        EXPORT   CoreFile=init.cgns  AscFile=init.tif
        IMPORT   ModelFile=pn.cgns
        IMPORT   CoreFile=pn.cgns
\end{verbatim}

\subsubsection*{Hint}
VTK file is intended to be used for post process. User can use
Paraview\footnote{\href{http://www.paraview.org}{http://www.paraview.org}},
MayaVi or
VisIt\footnote{\href{http://www.llnl.gov/visit}{http://www.llnl.gov/visit}}
to open and view VTK file. Further more, CGNS file is also supported
by VisIt.


\newpage
\section{Post Process}
\subsection{Plot}
The \textbf{PLOT} statement initializes the graphical display device for two and three dimensional
plots of device characteristics(3D) and device meshes(2D).

\subsubsection*{Syntax}
\begin{verbatim}
     PlotMesh  [TIFF.Out=<s>]
     Plot  Variable=Mesh [PS.Out=<s>] [TIFF.Out=<s>]
           [Resolution=(RES.Low|RES.Middle|RES.High)]
     Plot  Variable=(Na|Nd|ElecDensity|HoleDensity|Potential|EFieldX|EFieldy|Temperature)
           [ Measure=(Linear|SignedLog) ]
           [ PS.Out=<s> ] [ TIFF.Out=<s> ] [ Resolution=(RES.Low|RES.Middle|RES.High) ]
           [ AzAngle=<n> ] [ ElAngle=<n> ] [ Style=(Scale|Color|GrayLevel) ]
\end{verbatim}

\small
\noindent\begin{longtable}{ccccp{7cm}}
\textbf{parameter}   & \textbf{type}    & \textbf{default} & \textbf{unit} & \textbf{description} \\
Variable      & string  & -  & -             & This parameter specifies plot context.\\
PS.OUT        & string  & -  & -             & Specifies the postscript file name. The plot window will be saved to it.\\
TIFF.OUT      & string  & -  & -             & Specifies the TIFF file name. The plot window will be saved to it.
                                               Only available for X11 system. \\
Resolution    & string  & RES.Middle  & -    & The resolution of plot window.\\
Measure       & string  & Linear &           & Specifies the data axis to be linear or logarithmic.  \\
AzAngle       & number  & 240 & $\mathrm{degree}$  & The initial azimuthal rotation angle, 0$\leq$\textbf{AzAngle}<360. \\
ElAngle       & number  & 60  & $\mathrm{degree}$  & The initial elevation rotation angle. 0$\leq$\textbf{ElAngle}<70.\\
Style         & string  & Color   & -              & The plot style. \\
\end{longtable}
\normalsize

\subsubsection*{Example}
\begin{verbatim}
        PLOT  Variable=Mesh PS.OUT=mesh.ps
        PLOT  Variable=Nd Resolution=RES.High AzAngle=240  ElAngle=40  Style=Scale
        PLOT  Variable=Potential  Resolution=RES.Middle TIFF.out=potential.tiff\
              AzAngle=240  ElAngle=40  Style=Color
\end{verbatim}

\subsubsection*{Hint}
\textbf{PlotMesh} is an interactive GUI for mesh display only exist for X11 system.

The \textbf{PLOT} command can be used on both X11 and Win32 systems.
The 3D plot can be rotated by mouse and terminated by ESC key press.
If PS.OUT or TIFF.OUT argument is specified, the latest window image will be saved.


\newpage
\subsection{Probe}
The \textbf{PROBE} statement is used to extract field data along a
user defined segment. The segment can be a boundary or a segment
pre-defined in the region. For the in-region segment, GSS will set
it as Neumann boundary with no heat flux which takes not effect to
simulation result.

\subsubsection*{Syntax}
\begin{verbatim}
     Probe  Variable=(Na|Nd|ElecDensity|HoleDensity|Potential|EFieldX|EFieldy|Temperature)
            Region=<s> Segment=<s> ProbeFile=<s> Append=<b>
\end{verbatim}

\small \noindent\begin{longtable}{ccccp{9cm}}
\textbf{parameter}   & \textbf{type}    & \textbf{default} & \textbf{unit} & \textbf{description} \\
Variable      & string  & -  & -             & This parameter specifies probe context.\\
Region        & string  & -  & -             & Specifies the region name which the segment belongs to.\\
Segment       & string  & -  & -             & Specifies the segment name for probing the data.\\
ProbeFile     & string  & -  & -             & The file name for recording the data.\\
Append        & bool    &OFF & -             & Specifies the data should be appended to the file.  \\
\end{longtable}
\normalsize

\subsubsection*{Example}
\begin{verbatim}
    PROBE  Variable=Nd         Region=Si Segment=PB1 ProbeFile=Nd.txt Append=Off
    PROBE  Variable=Potential  Region=Si Segment=PB2 ProbeFile=P.txt  Append=On
\end{verbatim}

\subsubsection*{Hint}
Each \textbf{PROBE} statement records one variable for the whole segment to a user-specified file.
GSS pushes \textbf{PROBE} statement as the sequence of input text into a stack until a
\textbf{SOLVE} statement is met. These \textbf{PROBE} statements record the data during solve process.
After that, GSS will clear the stack. In short, \textbf{PROBE} only operates for next \textbf{SOLVE} process.

The file format for probe is show as follows:
\begin{verbatim}
    #region_name segment_name
    #node_num  X   Y
    #  0       x0  y0
    #  1       x1  y1
    #  ..............
    #SOLVE_TYPE  variable_name
    [V/I/Time]       v0     v1     v2 ...
    .....................................
\end{verbatim}
The head of file is the segment information, including region name,
segment name, total node number and the location of each node. The
last line of head shows the solve type and variable name. The solve
type can be "EQUILIBRIUM", "STEADYSTATE", "DCSWEEP\_VSCAN",
"DCSWEEP\_ISCAN" and "TRANSIENT". For the last three types, GSS will
record V/I/Time in the first column, respectively. The variable
value for all the nodes are listed in the same line.

%\begin{figure}[ht]
%\centering
%\includegraphics[scale=0.2]{plot1.png}
%\caption{2D and 3D plot.}
%\end{figure}


\newpage
\section{Convergence Problem}
The core arithmetic of GSS is solving the large scale nonlinear
equations arisen from semiconductor drift-diffusion model by
Newton's Iterative method. There are three factors which affect the
convergence of nonlinear solvers: the initial value, the Jacobian
Matrix and the inner linear solver. One must ensure that the initial
value is sufficiently near the real solution, the Jacobian Matrix is
exact or at least nearly exact and the inner linear solver can give
a suitable solution. When one of the three demands is not satisfied,
the convergence problem may raise. However, several skills can help
convergence.

If the first time running failed due to bad initial value, one can
employ a transient solver to do a time evolved solution. Set time
step to a few ps, and the solution on every step may get
convergence. After some certain steps, the initial shock is damped
and physical variables are forced to get close to real quantities.
Then the steady-state solver may work and you can get the
equilibrium solution.

Since version 0.46, GSS use automatically differentiation to
calculate Jacobian matrix. In most situations, author can guarantee
that Jacobian matrix is exact, except some rigorous situations when
round-off error is un-neglectable. However, GSS offers alternative
choice, the Matrix-Free method to set Jacobian Matrix by finite
difference approximation. This choice can be invoked with the
command line option \textit{-snes\_mf\_operator}. The Matrix-Free
method works well when impact ionization takes place, but it runs
much slower than original method.

Sometimes the LineSearch method may failed due to bad search
direction. If one get divergence message during DC sweep and
transient simulation when using LineSearch method, one can try
TrustRegion or basic Newton method.

Newton damping is a powerful tool to help convergence. It can work
with LineSearch and Basic Newton solvers. GSS has two damping
method, BankRose and Potential. Usually, damping Potential is better
than BankRose.

The most difficult problem is the failure of inner linear solver.
When the Jacobian Matrix is singular, problems may happen.
Especially one sets electrodes with lumped resistor or current
sources. If one get a convergence failed message for these
situations, please check the problem by adding command line option
\textit{-ksp\_monitor} to exam the convergence history. For more
information, one can use \textit{-ksp\_singmonitor} to get the
condition number of matrix (this works only with GMRES method).

The author suggests some method to overcome the problem. First, one
may improve the condition number by enlarging the DopingScale, but
this will increase numerical error. Second, one should carefully
choose the linear solver.

Here is the introduction of the main linear solvers GSS can use.
GMRES is a robust method for non-symmetric matrices. It must retain
all the previous vectors during iterative. The implemented code
often uses a "restart" method to avoid large memory requirement.
Sometimes the solution breaks when restart too often. One can
increase this restart steps by \textit{-ksp\_gmres\_restart <n>} (n
is the restart steps, default 150) BiCG and CGS often have irregular
convergence behavior. The irregular result may get things worse.
Bi-CGSTAB is the improved method to BiCG and CGS, which avoids the
irregular convergence patterns of BiCG/CGS while maintaining about
the same speed of convergence. TFQMR avoids the irregular
convergence behavior of BiCG. Also it avoids some breakdown
situations of BiCG. When BiCG temporarily stagnates or diverges,
TFQMR may still works. At last, LU factorization is the basic method
for solving linear systems. Besides build-in LU solver, PETSC can be
compiled with external LU factorization package such as SuperLU and
UMFPACK. This method works slow but usually more stable than
iteration solver.

In conclusion, LU factorization is recommend for conquering the
singular problem. But user can try GMRES with large restart steps,
Bi-CGSTAB and TFQMR methods for better efficiency.


\section{Memory and CPU requirement}
Thanks to C++'s dynamic memory manege system, GSS can solve problems
with any scale (at least, theoretically). The memory requirement is
not a serious bottleneck. A very large problem which contains 100K
nodes only requires about 300MB memory. This requirement is easy to
be satisfied with modern computers.

Because the core arithmetic of GSS is solving nonlinear equations,
which involves lots of solutions of linear system, the CPU time is
related with linear solvers, which is $O(n^3)$ with LU solver and
$O(n^2)$ with krylov iterative solver, in which $n$ is the problem
scale.

Fig\ref{CPUTime1} shows the CPU time vs node number with a PN diode
simulation by BCGS method on a Xeon 3.6GHz workstation. The time is
approximate the square of problem scale. It only requires serial
seconds when total node number less than 5000. But CPU time raises
to some minutes when node's number reach to 100K.

\begin{figure}[tbh]
\centering
\includegraphics[scale=0.2]{CPU.png}
\caption{CPU time vs problem scale} \label{CPUTime1}
\end{figure}

\appendix
\begingroup
  \makeatletter
  \let\chapter=\section
  % subsections goes into bookmarks but not toc
  \hypersetup{bookmarksopenlevel=1}
  \addtocontents{toc}{\protect\setcounter{tocdepth}{1}}
  % The \section command acts as \subsection.
  % Additionally the title is converted to lowercase except
  % for the first letter.
  \def\section{%
    \let\section\lc@subsection
    \lc@subsection
  }
  \def\lc@subsection{%
    \@ifstar{\def\mystar{*}\lc@sec}%
            {\let\mystar\@empty\lc@sec}%
  }
  \def\lc@sec#1{%
    \lc@@sec#1\@nil
  }
  \def\lc@@sec#1#2\@nil{%
    \begingroup
      \def\a{#1}%
      \lowercase{%
        \edef\x{\endgroup
          \noexpand\subsection\mystar{\a#2}%
        }%
      }%
    \x
  }
  \include{fdl}
\endgroup

\end{document}
